% !TEX root = ../thesis.tex

\chapter{Background}
\label{ch:background}

In this chapter we describe existing technology and concepts that will be referred to throughout the paper.

\section{GitHub}

QuickFeed already relies on GitHub to manage assignments and student code.
This section looks at certain GitHub features that we want QuickFeed to support.

\subsection{Issues}

GitHub issues is a useful feature for tracking, discussing and logging various issues/problems on a GitHub repository.
Any collaborator to a repository may open a GitHub issue, and describe any problem, idea or issue they have.
Other users may then contribute to the issue by commenting on it; thus easing the communication process within a project.
GitHub issues can therefore function as a discussion hub, and makes it especially useful for larger projects involving several people. 

\begin{figure}[ht]
    \centering
    \includegraphics[width=\textwidth]{photos/github-issue.PNG}
    \caption{Example of a GitHub issue comment section}
    \label{fig:github-issue}
\end{figure}

\subsection{Pull Requests}

Pull requests are a desire to merge any feature branch, into the main branch of a git repository.
GitHub supports managing pull requests through its user interface, and allows any contributor to a repository to create a pull request.
Through the user interface, progress on a branch can be tracked, reviewed and commented on.
In this sense, GitHub pull requests function as a central hub for feature branches.

Code review is also a central part of the pull request process.
Any eligible user may review, comment on, and request changes to the source code of a given pull request.
The reviewer may also use reviews to approve a pull request for merging.

Pull requests can also be linked to issues.
Doing this will automatically associate any linked issue to the pull request, and cause them to close when the pull request closes.
A common workflow would be to create an issue, describing a problem, and then creating an associated pull request for this issue.

Through its features, GitHub pull requests provide an efficient and manageable way of implementing new features to any project.

\begin{figure}[ht]
    \centering
    \includegraphics[width=\textwidth]{photos/pull-request.PNG}
    \caption{Example of a GitHub pull request}
    \label{fig:pull-request}
\end{figure}

\subsection{Workflows}

Maybe maybe

\section{QuickFeed}

QuickFeed provides two primary features.
A web interface that allows students to enroll in courses, create groups and receive feedback on their submissions.
And a backend that tests, scores and grades student submitted code.
The backend is primarily implemented with the Go programming language, while the frontend uses TypeScript and the React library.
For data storage, QuickFeed relies on an SqLite-database, managed using the the GORM library.

In this section we give a description of some of the processes and features QuickFeed supports.

\subsection{Protocol Buffers}

Protocol buffers is a language-neutral mechanism for serializing structured data. % (ref: https://developers.google.com/protocol-buffers)
It is often abbreviated as protobuf, and is used to define messages in a \textit{.proto} file.
When the file is compiled, data structures and methods for the desired language are generated in a separate file.
QuickFeed uses protobuf to generate most of its data structures.
These are then stored in its internal database when necessary.

\lstinputlisting[caption={Repository message}, firstline=90, lastline=107]{code/ag.proto}

When compiled, the above message generates the \textit{Repository} data structure, which in turn is used by QuickFeed to represent repositories.

It should be noted that QuickFeed also uses these messages in conjunction with gRPC to facilitate server-client communication.
This part is however not relevant for this project.

\subsection{Webhooks}
\label{sec:webhooks}

QuickFeed communicates with GitHub in two ways.
In this section we will detail one of them, namely webhooks.

A webhook is a "user-defined callback over HTTP". % (ref: (09.04) https://developer.atlassian.com/server/jira/platform/webhooks/)
In general, webhooks allow developers to listen to events from any supporting sites.
When any such event occurs, an HTTP request is sent to the address configured for the webhook.
The request contains data about the event, usually in a JSON format.

QuickFeed uses webhooks to retrieve data from push events on a course.
When a new course is created, QuickFeed creates a webhook on the GitHub organization of the given course.
This webhook is only triggered by push events, which are then handled by QuickFeed accordingly:

\begin{itemize}
    \item If the push event is to the tests repository, QuickFeed will update the course assignments.
    \item If the push event is to a student repository, QuickFeed will determine the assignments that have been changed/worked on, 
    and run the assignment tests on them.
\end{itemize}

% TODO: Should also maybe mention the payload that is sent.

\subsection{SCM API}
% TODO: Should say that QF authenticates with the GitHub API by being an Oath app.
The second way QuickFeed communicates with GitHub is through a custom API.

To facilitate communication with potentially any source control management system, a custom SCM API has been developed for QuickFeed.
The current iteration of QuickFeed supports interacting with GitHub through this API, via the go-github library. % (ref: go-github)
The library itself, communicates with GitHub's own REST API using HTTP requests.

The SCM API allows QuickFeed to perform various tasks, such as creating/updating repositories, creating webhooks, managing teams and more.
It, together with webhooks, form a 2-way communication stream between QuickFeed and GitHub.

\subsection{QuickFeed Repository Structure}
\label{sec:quickfeed-repository-structure}

When a teacher creates a course in QuickFeed, a GitHub organization is created to represent it.
Within this organization, three initial repositories are created, as well as student repositories when students enroll.
They are all described as follows.

\textit{info}: This repository simply holds information about a course.
The repository is available for all students, and would typically work as a simple information hub.
It is created and managed by the teaching staff.

\textit{assignments}: The repository responsible for presenting the assignments to students.
Every assignment in a course is represented as a folder within this repository.
As teachers push new assignments to this repository, or update existing ones, students pull the changes to their own local git repositories.

\textit{tests}: Within this repository, every assignment is also represented as a distinct folder.
The repository differs from \textit{assignments}, because every assignment folder contains a \textit{assignment.yml} file and test code.
In addition, the repository can contain a \textit{scripts} folder, containing a \textit{run.sh} file and a Dockerfile.

The \textit{assignment.yml} file is used by teachers to specify assignment specific settings.
As an example, it can contain information about the assignment deadline, whether it is manually graded, and more.

The \textit{run.sh} file contains the script used by QuickFeed when testing student code.
As explained, it is contained within the \textit{scripts} folder, but can also be located within the specific assignments themselves.
When running tests on an assignment, QuickFeed will prioritize using any assignment specific script files.

Finally, the Dockerfile contained within \textit{scripts} is used to create a docker image.
Within this image, QuickFeed runs the appropriate script, derived from \textit{run.sh}.

Being a repository that is used and managed solely by the teaching staff, it follows that students do not have access to its contents.

\begin{figure}[ht]
    \centering
    \includegraphics[scale=0.8]{photos/tests-repository-structure.PNG}
    \caption{Example of a tests repository folder structure}
    \label{fig:tests-repository-structure}
\end{figure}

Student repositories: There are two types of student repositories, user and group repositories.
A user in this case would simply be any student who has enrolled in the course.
Whereas a group is any number of students that are working together.
Naturally, student repositories are only accessible by the students associated with them, and the teaching staff.

Student repositories are named according to the following format: \textit{name-labs}.
For an enrolled student, \textit{name} would be their GitHub user name.
A group's \textit{name} however, would be defined by the group members themselves when they enroll the group.

These repositories are the ones students push their code to as they work on assignments.

\begin{figure}[ht]
    \centering
    \includegraphics[width=\textwidth]{photos/qf-repository-structure.png}
    \caption{GitHub repository structure for a QuickFeed course}
    \label{fig:qf-repository-structure}
\end{figure}

\subsection{The Score Package}

QuickFeed's score package allows for scoring student submitted code.
Every time a student pushes code to their repository, QuickFeed will run tests on the code, and generate a total score ranging from 0 to 100.
In this section we describe the parts of this package that are relevant for this thesis.

When teachers develop assignments, they also create tests that are supposed to test student code.
As part of the score package, teachers can specify for individual tests, how much score they should give, and how that score is gained.
For example, a test can loop through several test conditions, and then decrement the score every time it fails.
These tests have to be explicitly added by teachers using either Add or AddSub, both of which are functions defined in the package.
Doing so, they are added to the pool of all tests that constitute an assignment.
A total score is then generated on the server side by adding up the score from all added tests.
This process is further explored in the next section.

\subsection{Testing Assignments}

So far we have mentioned that students push code to their respective GitHub repositories, and that QuickFeed will run assignment specific tests on this code as this happens.
In this section we further explain the parts of this process that are important to this thesis.

The process starts by teachers specifying individual tests an assignments, 

need:
how teachers add functions as part of an assignment test
how QF will score based on these assignments

how QF creates a docker container and runs the specific script described in previous section
how this script file uses arguments to clone user repositories
QF then extracts the result from the tests run