%!TEX root = ../thesis.tex

\chapter{Discussion}
\label{ch:discussion}

There is nothing stopping a student to ignore everything we have done, and simply solve an assignment manually.
I.e, there is nothing enforcing a PR review.

What happens if a teacher creates two tasks with the same name?

\section{Future Work}

Maybe more feedback if students do something wrong.
For example, if a student creates a pull request without correctly linking the issue, they should probably be immediately made aware.
In this case, we could maybe comment on their pull request, stating this.
We could also support students linking issues late by listening to pull request body changed events, or something.
This avoids the need for students to recreate pull requests in case they fail.

\subsection{Checks API}

\subsection{Expanding Approval Process}
Pull requests can be approved. This has no impact on the assignment approval process. Can expand on it.

\subsection{GitHub App}
\label{section:github-app}

QF should probably be a GitHub app instead of Oauth app
Oauth app acts on behalf of a user, while app uses its own identity.
For example, with an Oauth app we cannot create pull requests and comments as QF, only as an authorized user in that org.
All issues that are created will be in the course creators name.
Also required for Checks API.
https://docs.github.com/en/developers/apps/getting-started-with-apps/differences-between-github-apps-and-oauth-apps
https://docs.github.com/en/developers/apps/getting-started-with-apps/about-apps

\subsection{Instructions}

Should probably include a chapter somewhere that describes all functionality.
Such as how to restore a working state if a student merges a pr too early.