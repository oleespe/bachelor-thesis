%!TEX root = ../thesis.tex

\chapter{Introduction}
\label{ch:intro}

The University of Stavanger has for the last couple of years been developing a project known as QuickFeed.
Since its inception in 2015, both teachers and students have contributed to its development. % (ref)

QuickFeed aims to ease the process of submitting student assignments for both students and the teaching staff.
Though not exclusively, the project is primarily developed with the intent of handling code based assignments.
QuickFeed will run predetermined tests on submitted code, and will use their results to provide feedback for students via its web interface.
Using QuickFeed, assignments can be both scored and graded without the teaching staff ever having to intervene.

To accommodate these features, QuickFeed uses GitHub to, amongst other things, manage student code.
GitHub is therefore one of the primary sites students interact with, when working on QuickFeed managed assignments.
Because of this, there is a desire to further integrate QuickFeed with GitHub.
Specifically, we aim to use GitHub's pull request and issue features to give students another avenue for manual and automatic feedback.

\section{Motivation and Background}
\label{sec:motivation}

In the current iteration of QuickFeed, students only receive automated feedback on their code through QuickFeed's custom web interface.
Here, they can see what parts of their code is failing, and also whether they have successfully passed the assignment.
The motivation behind this thesis is to further expand on how students receive feedback when working on an assignment.

Using GitHub pull requests we can facilitate manual feedback directly on student code by using its review features.
Pull requests also have the potential to support automatic feedback, e.g. by another GitHub feature called workflows.

\section{Objectives}

This project has the two following objectives.

To provide automated and manual feedback on student code, QuickFeed will support using GitHub pull requests.
Facilitating manual feedback means QuickFeed will decide when to assign reviewers to a pull request and who to assign.
When a pull request is approved, QuickFeed will also determine if this approval was legitimate.
To support automated feedback via pull requests, we want to use GitHub workflows, and have them work in conjunction with QuickFeed.

To further accommodate pull requests, we want QuickFeed to support task markdown files within an assignment.
These will function as a way for teachers to subdivide their assignment into smaller parts.
For each task, a GitHub issue will be created on every student repository by QuickFeed.
These issues are the ones students will base their pull request on.

\section{Approach and Contributions}

\begin{itemize}
\item Give a brief summary of your overall approach.
\item Summarize the specific contributions that you made in this thesis (implementation, empirical results, analysis, etc.).
\item It might be presented as a bullet list.
\end{itemize}


\section{Outline}

\begin{itemize}
\item Give an overview of the main points and the structure of your thesis.
\item Examples: ``Chapter 2 covers ...  Chapter 3 describes ...''
\item Show how the different parts (chapters) relate to each other.
\end{itemize}
