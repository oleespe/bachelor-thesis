%!TEX root = ../thesis.tex

\chapter{Introduction}
\label{ch:intro}

The University of Stavanger has for the last couple of years been developing a project known as QuickFeed.
Originally developed in 2015, the project has since then been through numerous changes. % (ref)
QuickFeed has since its inception been used by several computer science classes at the university.

QuickFeed aims to ease the process of managing student assignments, and does so for both students and the teaching staff.
Though not exclusively, the project is primarily developed with the intent of handling code based assignments.
QuickFeed will run predetermined tests on submitted code, and instantly provide feedback back to students through its web interface.
Using QuickFeed, assignments can be both scored and graded without the teaching staff ever having to intervene.

To accommodate these features, QuickFeed uses GitHub to, amongst other things, manage student code.
GitHub is therefore one of the primary sites students interact with, when working on QuickFeed managed assignments.
Because of this, there has been a desire to further integrate QuickFeed with GitHub.
Specifically, with the issues, pull request and workflow features the service provides.

\section{Background and Motivation}
\label{sec:motivation}

The motivation for this project is to have GitHub issues, pull requests and workflows used when students are working on assignments.
Given that QuickFeed already heavily uses GitHub, it seems logical and advantageous to use more of the latter's features than we currently do.

More specifically, there is a desire to use GitHub pull requests as a hub for students.
If integrated with QuickFeed, pull requests can further expand on how students receive feedback, through the use of GitHub workflows.
Feedback can also be provided within a pull request through code review, either by other co-students or by the teaching staff.

This project is a continuation of the work done by Adil Khurshid. % (ref)
The motivation behind Adil's work, is the same as for this thesis.
His work was however not finished, which is why it is continued here.

\section{Objectives}

The objectives of the project are described as follows:
% This is maybe too descriptive, and parts of it are explained in other locations.
\begin{itemize}
    \item Teachers should be able to separate individual assignments into smaller tasks.
    \item QuickFeed should create an issue for each task on student repositories.
    \item Students should be able to create pull requests based on these issues.
    \item QuickFeed should assign both student and teacher reviewers to them.
    \item QuickFeed should use GitHub workflows to display some sort of feedback, based on the tests run.
\end{itemize}

\section{Approach and Contributions}

\begin{itemize}
\item Give a brief summary of your overall approach.
\item Summarize the specific contributions that you made in this thesis (implementation, empirical results, analysis, etc.).
\item It might be presented as a bullet list.
\end{itemize}


\section{Outline}

\begin{itemize}
\item Give an overview of the main points and the structure of your thesis.
\item Examples: ``Chapter 2 covers ...  Chapter 3 describes ...''
\item Show how the different parts (chapters) relate to each other.
\end{itemize}
