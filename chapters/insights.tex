%!TEX root = ../thesis.tex

\chapter{Insights}

\section{Results}

% TODO: Discussions on the implementation and how the project went.

\subsection{Manually Graded Assignments}

% TODO: Need to explicitly specify the difference between manually graded and approved.
Post implementation, a problem was encountered with how QuickFeed treats manually graded assignment.

To QuickFeed, a manually graded assignment is any assignment whose reviewers field is higher than zero.
When handling push events for such an assignment, no automated tests are run.
No automated tests, means no test scores, and also no assignment total score.

In our implementation we use scores for two purposes.
One is to automatically publish feedback on the pull request in question, while the other is to determine when to assign reviewers to a pull request.
If an assignment generates no scores, these pull requests would therefore be somewhat useless.
It seems then that any assignment taking advantage of the features developed in this project cannot be manually graded.

\subsection{Known Weaknesses}

% TODO: Write
There is a weakness in how we handle branches.
If someone pushes to a branch with a remote name that differs from the local name, we cannot know which local branch to check out.

What happens if a teacher creates two tasks with the same name?

Students can delete issues manually.
Students can also delete the pull request comment.

\section{Recommended Improvements}

Certain parts of the implementation has the potential to be improved.
These improvements primarily revolve around already implemented code, and are generally minor in scale.
They are therefore discussed here, instead of in the Future Work chapter, which primarily proposes larger scale projects.

\subsection{GitHub App}
\label{section:github-app}

When discussing how to create issues and pull requests in sections \ref{sec:issue-creation} and \ref{sec:creating-pull-requests} respectively, we explored the need for QuickFeed to be able to identify as "itself".
Implementing QuickFeed as a GitHub app would accomplish this, as a GitHub app acts on its own behalf when interacting with the GitHub API \cite{apps}.

Running concurrently with this project, was another that converted QuickFeed into a GitHub app.
For this implementation to be fully optimal, it should take advantage of this fact.
For instance, creating issues should be done as QuickFeed, and not as students.
Similarly, while creating pull request feedback comments as students is not that problematic, it would be best if they too were created as QuickFeed.

\subsection{Task Naming Format}

A problem with the current implementation, is the fact that there are two types of task names.

As discussed in Section \ref{sec:tasks-and-issues-data-structures}, the task data structure has a field name.
It is set upon its creation, and is a combination of the assignment it is for, and the name of the markdown file that describes it.
For example, a task can have the name assignment1/hello\_world.
When complementing the score package to also support tasks, we instead use a local task name, i.e., only hello\_world.

Originally, including the assignment name was necessary to associate tasks with each other when synchronizing them.
Since then, the implementation has changed, and including the assignment name is no longer needed.
A recommendation is therefore to make the local task name standard, thereby avoiding any unwanted confusion.

\subsection{User Experience}

% TODO: Write
Maybe more feedback if students do something wrong.
For example, if a student creates a pull request without correctly linking the issue, they should probably be immediately made aware.
In this case, we could maybe comment on their pull request, stating this.
We could also support students linking issues late by listening to pull request body changed events, or something.
This avoids the need for students to recreate pull requests in case they fail.

Should discuss that finding the linked issue by parsing the PR body limits what can be in it.
Can maybe instead be implemented as a regular expression search.

\subsection{QuickFeed Test Environment}

% TODO: Write
A struggle throughout the course of the entire project, was facilitating testing of the features that were developed.
QuickFeed does not have an efficient way of testing.
As example, to test the process of 

Brief explanation on existing testing environment, i.e., qtest
For example, we have two populateDatabaseWithTasks functions in two different places

Can maybe discuss the potential to have a more efficient way of testing things, such as webhook related stuff.
For example, I am currently hard coded into some of the db population functions, to facilitate testing.
Would be better if this was more independent. 
Could maybe have certain QF test GitHub users?