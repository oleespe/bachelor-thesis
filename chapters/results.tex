%!TEX root = ../thesis.tex

\chapter{Results}

\section{Insights}

% TODO: Discussions on the implementation and how the project went.

\subsection{QuickFeed Test Environment}

% TODO: Write
Brief explanation on existing testing environment, i.e., qtest
For example, we have two populateDatabaseWithTasks functions in two different places

Can maybe discuss the potential to have a more efficient way of testing things, such as webhook related stuff.
For example, I am currently hard coded into some of the db population functions, to facilitate testing.
Would be better if this was more independent. 
Could maybe have certain QF test GitHub users?

\subsection{User Experience}

The features implemented in this project impose new demands on both students and teachers wanting to use them.
The most apparent of them are listed in this section.

Students solving group assignments with tasks, must be able to, or made aware of the following:
\begin{itemize}
    \item How GitHub issues work, and the fact that they are representations of tasks that teachers have created.
    \item How to create new branches and commit to them.
    \item How to create pull requests from these branches, and link the correct issue to them.
    \item How to actually solve tasks.
    Meaning that they must know which parts of an assignment is relevant for a given task.
    \item How the general flow of an assignment with tasks goes.
    In essence, they must know the three stages that are relevant for pull requests; draft, review and approved.
    \item How to close and merge pull requests when they are approved.
    \item How to handle situations where something goes wrong.
    For example, if they create a pull request without correctly linking the relevant issue.
\end{itemize}

Similarly, teachers and the teaching staff in general must know the following:
\begin{itemize}
    \item How to create tasks via markdown files.
    \item How to associate tests with tasks, using the score package.
    \item How to help students failing with any of the points in the previous list.
\end{itemize}

Especially for students, there is a lot of new "stuff" to deal with.
Any assignment would have to carefully explain the points mentioned above.

\section{Known Limitations}

There are certain known limitations to this project's implementation.
They are listed in this section.

\subsection{Differing Branch Names}

There is a weakness related to students potentially having remote and local branches with different names.

To facilitate testing and scoring of student code on non-default local branches, we have to check out that local branch.
We get the branch name to check out, through the pull request opened event payload.
This branch name however, is for the remote GitHub repository the pull request was created for.
Students can in theory create a local branch called "branch1", and push it to the remote branch "branch2".
If this is done, QuickFeed has no reference to the local branch it needs to run tests on.

\subsection{Human Error}

There are certain actions students can perform, that QuickFeed simply cannot handle with the current implementation.

For instance, if students were to manually delete the issues that QuickFeed creates, there would be no way of recreating them.
This would break the entire assignment process, as students can no longer create pull requests for these issues.
Similarly, students can also delete the feedback comments created on their pull requests.
If this happens, an error will occur when QuickFeed tries to publish new feedback to the deleted comment.

Of course, these are fringe cases, as they require students doing something the probably would know to be incorrect.
They are still worth taking into consideration.

\subsection{Manually Graded Assignments}

% TODO: Need to explicitly specify the difference between manually graded and manually approved.
Post implementation, a problem was encountered with how QuickFeed treats manually graded assignment.

To QuickFeed, a manually graded assignment is any assignment whose reviewers field is higher than zero.
When handling push events for such an assignment, no automated tests are run.
No automated tests, means no test scores, and also no assignment total score.

In our implementation we use scores for two purposes.
One is to automatically publish feedback on the pull request in question, while the other is to determine when to assign reviewers to a pull request.
If an assignment generates no scores, these pull requests would therefore be somewhat useless.
It seems then that any assignment taking advantage of the features developed in this project cannot be manually graded.

\section{Possible Improvements}

Certain parts of the implementation has the potential to be improved.
These improvements primarily revolve around already implemented code, and are relatively minor in scale.
They are therefore discussed here, and any larger scale project is instead proposed in Chapter \ref{ch:future-work}.

\subsection{GitHub App}
\label{section:github-app}

When discussing how to create issues and pull requests in sections \ref{sec:issue-creation} and \ref{sec:creating-pull-requests} respectively, we explored the need for QuickFeed to be able to identify as "itself".
Implementing QuickFeed as a GitHub App would accomplish this, as a GitHub App acts on its own behalf when interacting with the GitHub API \cite{apps}.

Running concurrently with this project, was another that converted QuickFeed into a GitHub App.
For this implementation to be fully optimal, it should take advantage of this fact.
For instance, creating issues should be done as QuickFeed, and not as students.
Similarly, while creating pull request feedback comments as students is not that problematic, it would be best if they too were created as QuickFeed.

QuickFeed being a GitHub App, also opens the door for pull requests being created automatically, as discussed in Section \ref{sec:creating-pull-requests}.
Any implementation would still have to solve the problem of deciding the owner of a given pull request.
To clarify, even if pull requests are created anonymously, QuickFeed still needs to know which student is actually working on it.
Otherwise, we have no way of knowing from which pool of group members we should assign reviewers, as explained in Section \ref{sec:assigning-reviewers}.

\subsection{Task Naming Format}

A problem with the current implementation, is the fact that there are two types of task names.

As discussed in Section \ref{sec:tasks-and-issues-data-structures}, the task data structure has a field name.
It is set upon its creation, and is a combination of the assignment it is for, and the name of the markdown file that describes it.
For example, a task can have the name assignment1/hello\_world.
When complementing the score package to also support tasks, we instead use a local task name, i.e., only hello\_world.

Originally, including the assignment name was necessary to associate tasks with each other when synchronizing them.
Since then, the implementation has changed, and including the assignment name is no longer needed.
A recommendation is therefore to make the local task name standard, thereby avoiding any unwanted confusion.

\subsection{Linking Issues}

When students create pull requests, a requirement is that they also link the relevant issue, by inserting a reference to them in the pull request body.
The implementation for this has two problems.
First of, the function that parses the issue from the pull request body, described in Code \ref{code:getLinkedIssue}, is limited in its capabilities, and demands a certain format.
Secondly, if students were to fail, when linking pull requests to issues, they are not immediately made aware.

To link an issue to a pull request, GitHub supports using a keyword such as "Fixes" or "Closes", followed by the character sequence "\#<issue number>".
As an example, linking issue \#30 to a new pull request, can be done by inserting "Fixes \#30" in the pull request body.
The function that parses the issue number from this sequence is limited to only handling simple cases, as it simply splits the pull request body.
A better solution would probably be to use a regular expression search.

To tackle the second problem, a solution could be to insert a comment or something similar in the faulty pull request, stating that an associated issue could not be found.
Another approach could be to support linking issues after the pull request has already been created.
Thereby removing the need for students to create a completely new pull request.

\subsection{TITLE}

% TODO: Should discuss how it would be easy to use all features except student code review on all student repositories.
% Should maybe be in future work.