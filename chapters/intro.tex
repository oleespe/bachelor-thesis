%!TEX root = ../thesis.tex

\chapter{Introduction}
\label{ch:intro}

The University of Stavanger has for the last couple of years been developing a project known as QuickFeed.
Since its inception in 2015, both teachers and students have contributed to its development \cite{autograder, reactjs, redesign, grpc}.

QuickFeed aims to ease the process of submitting student assignments for both students and the teaching staff.
Though not exclusively, the project is primarily developed with the intent of handling code based assignments.
QuickFeed will run predetermined tests on submitted code, and uses their results to provide students with feedback via its web interface.
Using QuickFeed, assignments can be both scored and graded without the teaching staff ever having to intervene.

To accommodate these features, QuickFeed uses GitHub to, amongst other things, manage student code.
GitHub is therefore one of the primary sites students interact with, when working on QuickFeed managed assignments.
Because of this, there is a desire to further integrate QuickFeed with GitHub.
Specifically, we aim to use GitHub's pull request and issue features to give students another avenue for manual and automatic feedback.

\section{Motivation}
\label{sec:motivation}

In the current iteration of QuickFeed, students only receive automated feedback on their code through QuickFeed's custom web interface.
Here, they can see what parts of their code is failing, and also whether they have successfully passed the assignment.
The motivation behind this thesis is to further expand on how students receive feedback when working on an assignment.

Using GitHub pull requests we can facilitate manual feedback directly on student code, by using its review features.
Pull requests also have the potential to support automatic feedback, e.g., by another GitHub feature called workflows.

\section{Objectives}

This project has the following fundamental objectives.

First of, we want to give teachers the option of subdividing assignments into individual tasks, each describing specific problems students need to solve.
When a student wants to solve a given assignment task, they create a pull request, in which they implement all code relevant for that task.

Furthermore, in this project, we want QuickFeed to assign reviewers to these pull requests when appropriate.
These reviewers can request changes and recommend improvements where they deem necessary, thereby providing students with a new way of receiving feedback.

As students push code to their pull requests, we also want QuickFeed to publish some sort of automatic feedback, giving the students a general indication on how well they are doing, directly on the pull requests themselves.