\chapter{Future Work}

% TODO: Write intro
This is only a baseline.
Lots of potential for the features developed.

\section{Further QuickFeed Integration}

Even though QuickFeed now supports pull requests, issues and tasks, they are not highly integrated with the rest of QuickFeed.
The features developed in this project are mostly standalone, and not generally connected with QuickFeed's existing ones.

To illustrate, we can imagine a teacher creating an assignment with several tasks.
A group could in theory correctly solve all parts of this assignment, push it to their group repository, and then have it scored the regular way.
They would still get a passing score in QuickFeed's web interface, all the while having skipped the entire pull request process.
Similarly, doing things the intended way also has no direct consequence on how QuickFeed treats the assignment submission as a whole.
Having a pull request approved, has no impact on how an assignment is approved.
Only when a teacher approves the assignment via the web interface, is it actually approved.

A suggestion is therefore to further integrate the implemented features with QuickFeed's assignment submission process.
For example, a requirement to getting an assignment approved, could be to have all relevant pull requests approved.
To support this, one would have to develop a system that compares the number of approved pull requests, to the number of tasks for an assignment.
Implementing this would also require a rethink of the entire assignment approval process.

If such a system is developed, there is also the potential to further integrate it with QuickFeed's existing web interface.
For example, assignments that rely on pull request approval, can be made to not require explicit approval via the web interface.
The assignment can instead be automatically approved once all relevant pull requests are.
This way, the need for a custom web interface diminishes.

Another way to move away from a custom web interface to simply relying on GitHub's, could be to expand on the feedback given in the pull request comments.
Technically, all data that is used for the current automatic feedback system, is also available to be published in these comments.
Of course, there might be limitations to the amount of information a pull request comment can reliably express, but the potential is still there.

\section{Further GitHub Integration}

There is also the potential to further integrate with GitHub.

GitHub's Checks API seems to have a lot of potential \cite{checks}.
In short, it can be used to create custom checks on a pull request, by sending special webhook events every time someone pushes code to a repository.
These checks can then be used on a pull request to signals whether it has been correctly approved or not
Furthermore, if we only allow pull request merges when this check passes, we can avoid students incorrectly merging pull requests altogether.
GitHub supports limiting access like this by specifying rules on repository branches \cite{branches}.
