%!TEX root = ../thesis.tex

\chapter{Conclusions}
\label{ch:conclusion}

This thesis had the goal of using GitHub's issue and pull request features to further expand on how students receive feedback when working on assignments.
By implementing QuickFeed support for teachers subdividing assignments into tasks, we have created a new way for teachers to approach making assignments.
Furthermore, by expanding QuickFeed's existing score package, we now support testing and scoring tasks.
This allows us to provide automated feedback directly on pull requests, as well as manual feedback by teachers and co-students on student code.

We believe that both students and teachers can benefit from the features implemented in this project.
Giving students an avenue for direct feedback on their code through pull request code reviews, seems especially beneficial.
Furthermore, familiarising students with GitHub pull requests can be highly beneficial for them in the long run.

Even though these features are functional, there is still much that can be done to improve them.
In this sense, what has been accomplished in this project, should hopefully serve as a good baseline for future projects.
