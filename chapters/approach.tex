%!TEX root = ../thesis.tex

\chapter{Approach}
\label{ch:approach}

Text text
 
\section{Existing Approach}

The existing approach to this project is detailed in the report written by Adil.
In short, his approach is summarized as follows.

When a teacher creates an assignment, they have the option of creating any number of task markdown files.
For each of these markdown files, a single issue should be created on every student repository.
The title and body of these issues are defined by the contents of the corresponding task.
When a student wants to complete a task, they create a pull request for the associated issue.
After the deadline of the assignment has passed, QuickFeed randomly assigns co-student and teacher reviewers to the pull request. 
Only when the pull request has been approved by a teacher, should it be merged.

Adil discusses several challenges to this approach, and gives his proposed solutions.
Given that he did not finish his work however, many of these solutions have later proved inadequate or faulty.
In general, this thesis is independent of the aforementioned approach, and uses it primarily as a source of inspiration.

\section{Desired workflow}

This section describes in detail the desired student and teacher workflow, and is in essence what we want to achieve with this project.

Tasks will be handled as described in the previous section.
They are created by the teacher as markdown files within an assignment, and QuickFeed creates issues based on them.
When a student wants to complete any task, they create a new branch on their local repository.
Using this branch, they create a pull request on their student repository that is linked to the associated issue.

Upon pull request creation, QuickFeed assigns reviewers to the pull request.
At least one teacher is assigned to the pull request as well as a desired number of students.

At this stage, a student has an open pull request on their repository with reviewers assigned.
Tests are run on the student code by QuickFeed, and their results are displayed via GitHub workflows.
This way, students receive feedback on their work directly through GitHub.

When students are finished with a task, i.e., they receive a passing score from the test results, their code is ready to be reviewed.
Any assigned reviewers can at this point request changes to the source code.
When all changes are implemented, the teacher approves the pull request, allowing the student to merge it.

When all tasks have gone through this process, the assignment itself can be set as approved.

\section{Challenges and Possible Solutions}

In the previous section we looked at what exactly it is we are trying to accomplish. 
A number of challenges become apparent when trying to implement these new features however.
In this section we discuss these challenges, as well as how to possibly overcome them.

\subsection{Increased Complexity}

Introducing pull requests to students can prove challenging in itself.
Even though pull requests are not exactly a challenging feature to learn and use, it can still prove time consuming for certain students.
Even more so, if the students are already struggling with how to use git.
More time and effort from the teaching staff might have to be dedicated to assisting students with issues they may have.

This is not an issue that can simply be solved.
Every teacher will have to decide themselves if the benefits outweigh the increased complexity when using tasks and pull request with their assignments.

\subsection{Multiple branches}

Another issue arises from the fact that students will now be working in branches other than the main one.
In the current version of QuickFeed tests are only run on the main branch of any repository.
To support pull requests, QuickFeed will have to be expanded to support running tests on any branch.

\subsection{Access to Student Repositories}

A major problem is how to handle access, when dealing with student to student code review.
In order for any student to review the code of another, they would have to somehow be given access to said code.

A solution could be to use GitHub teams to grant students review access to other repositories.
This access can then be removed once the review process is finished, or when the assignment deadline is passed.
This solution does however lead to possible complications.
First of all, students having access to another random student's repository can be somewhat invasive.
Secondly it can lead to the following possible scenario:

Imagine we have two students: Student-1 and Student-2.
Student-2 feels they are finished with a task from assignment1, and therefore requests a review.
Student-1 is assigned to review it, and is therefore granted access, by QuickFeed, to Student-2's repository.
Student-2 is also diligent, and has already started working on assignment2.
In fact, Student-2 has nearly completed assignment2.
Student-1 however, has not even started on assignment2, but because they can now access Student-2's assignments, they receive valuable knowledge on how to solve it.

Of course, this would be highly problematic, and any implementation would have to avoid this type of problem.
One possible solution is to create a clone repository containing only the assignment or task that is to be reviewed. 
This way, any reviewing student will only have access to the assignment in question, and not the rest of that students repository.
This solution would therefore avoid the problem described in the above scenario.

A functional implementation of the above solution seems highly complex, as there are a myriad of complications that would have to be accounted for.
For one, if an assignment contains for example 6 tasks, it would mean that 6 copy-repositories would have to be created and managed by QuickFeed for each student.
Let us say that there are 50 students taking a course, this would then mean that 300 repositories in total are created.
Obviously this is not a feasible way to tackle this issue, and instead any such implementation should only create one copy-repository per student.

Even if we ignore this single issue, there is still the problem of how to manage the individual copy-repositories.
How do we ensure that things stay consistent?
How would we handle students?

\subsection{Assigning reviewers}

\subsection{Student feedback}

\subsection{Malicious Student Behavior}

\section{Analysis}

Analyse the challenges from previous chapter, and how we can overcome them.
What compromises have to be made?

\section{New Approach}

A new approach that is to be implemented. Got to this approach through discussions, and attempted implementations.

\section{Proposed Implementation}

How we actually plan to implement features.