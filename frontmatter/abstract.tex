%!TEX root = ../thesis.tex

\abstract

QuickFeed is an automatic evaluation system, developed at the University of Stavanger, used for grading/scoring student code submissions. 
When a student submits code to GitHub, QuickFeed will automatically run tests on the supplied code, and return useful feedback to the student in question.
This way, much time can be saved for both students and teachers, when handling submissions.

In this thesis we explore the possibility of utilizing GitHub's pull request and issue features, together with QuickFeed.
Pull requests, specifically, can be used by teachers or other students to easily review student code.
As a result, students will have an approachable hub for receiving feedback.
The utility of pull requests can be further expanded by using GitHub workflows to display the results of tests run on student code.

% TODO: Something better here
To accommodate these features, QuickFeed is expanded to handle them.
