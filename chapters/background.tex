% !TEX root = ../thesis.tex

\chapter{Background}
\label{ch:background}

In this chapter we describe existing technology and concepts that will be referred to throughout the paper.

\section{GitHub}

In this section we explore features of GitHub that are relevant to this thesis.

\subsection{Issues}

GitHub issues is a useful feature for tracking, discussing and logging various issues/problems on a GitHub repository.
Any collaborator to a repository may open a GitHub issue, and describe any problem, idea or issue they have.
Other users may then contribute to the issue by commenting on it, easing the communication process within a project.
GitHub issues can therefore function as a discussion hub, and makes it especially useful for larger projects involving several people. 

\begin{figure}[ht]
    \centering
    \includegraphics[width=\textwidth]{photos/github-issue.PNG}
    \caption{Example of a GitHub issue comment section}
    \label{fig:github-issue}
\end{figure}

\subsection{Pull Requests}

Pull requests are a desire to merge any feature branch, into the main branch of a repository.
GitHub supports managing pull requests through its user interface, and allows any contributor to a repository to create a pull request.
Through the user interface, progress on a branch can be tracked, reviewed and commented on.
In this sense, GitHub pull requests function as a central hub for feature branches.

Code review is also a central part of GitHub pull requests.
Any eligible user may review, comment on and request changes to the source code of a given pull request.
The reviewer may also use reviews to approve a pull request for merging.

Pull requests can also be linked to issues.
Doing this will automatically associate any linked issue to the pull request, and cause them to close when the pull request closes.
A common workflow would be to create an issue, describing a problem, and then creating an associated pull request for this issue.

Through its features, GitHub pull requests provide an efficient way of implementing new features to any project.
As well as easing the communication of desired changes to any source code.

\begin{figure}[ht]
    \centering
    \includegraphics[width=\textwidth]{photos/pull-request.PNG}
    \caption{Example of parts of a GitHub pull request}
    \label{fig:pull-request}
\end{figure}

\subsection{Workflows}

Maybe maybe

\section{QuickFeed}

QuickFeed provides two primary features.
A web interface used by both teachers and students, and a backend that tests, scores and grades student submitted code.

The web interface is implemented with the Typescript programming language, using the React library.
It is used by students to enroll in a course, create groups and receive feedback on their code.

The backend is primarily implemented using the Go programming language, and is divided into different modules, each with its own set of responsibilities.
It also features an SqLite-database, implemented with the GORM library.

In this section we give a description of some of QuickFeed's features that are important in this project.

\subsection{Protocol Buffers}

Protocol buffers is a language-neutral mechanism for serializing structured data. % (ref: https://developers.google.com/protocol-buffers)
It is often abbreviated as protobuf, and is used to define messages in a \textit{.proto} file.
When the file is compiled, data structures and methods for the desired language are generated in a separate file.

QuickFeed uses this method to generate the basis of nearly all its data structures, which are then stored in its internal database.

\lstinputlisting[caption={Repository message}, firstline=90, lastline=107]{code/ag.proto}

When compiled, this message generates the Repository data structure, which in turn is used to represent any QuickFeed managed repository.

\subsection{QuickFeed Repository Structure}

When a teacher creates a course in QuickFeed, a GitHub organization is created to represent it.
Within this organization, three initial repositories are created, as well as student repositories when students enroll.
They are all described as follows.

\textit{info}: This repository simply holds information about a course.
The repository is available for all students, and would typically work as a simple information hub.

\textit{assignments}: The repository responsible for representing the assignments themselves to students.
Every assignment in a course is represented as a folder within this repository.
As teachers push new assignments to this repository, or update existing ones, students pull the changes to their own local git repositories.

\textit{tests}: Within this repository every assignment is also represented as a distinct folder.
The repository differs from \textit{assignments}, because it also contains assignment information and test code.

Assignment information is described in a \textit{assignment.yml} file, and should be present in every assignment folder.
Content within this file is used when QuickFeed creates the assignment, and can contain settings such as the assignment deadline.

Test code is any code created by the teaching staff used to test the student submitted code.

Additionally, \textit{tests} may contain \textit{run.sh} script files in a separate \textit{scripts} folder, or within an assignment folder.
The \textit{scripts} folder can also contain a Dockerfile, used to create a test environment.
The test code, script files and Dockerfile each contribute to how tests are run on student code.
Even though QuickFeed supports running automatic tests on student code, it is not a must for QuickFeed to function.

Being a repository that is used and managed solely by the teaching staff, it follows that students do not have access to its contents.

\begin{figure}[ht]
    \centering
    \includegraphics[scale=0.8]{photos/tests-repository-structure.PNG}
    \caption{Example of a tests repository folder structure}
    \label{fig:tests-repository-structure}
\end{figure}

Student repositories: There are two types of student repositories, user and group repositories.
A user in this case, would simply be any student who has enrolled in the course, and a group is a multitude of students.
A group on the other hand, is any number of students that are working together, and are enrolled separately.
Naturally, student repositories are only accessible by the students associated with them, and the teaching staff.

Student repositories are named according to the following format: \textit{"name"-labs}.
For an enrolled student, \textit{name} would be their GitHub user name.
A group's \textit{name} however, would be defined by the group members themselves when they enroll the group.

These repositories are the ones students push their code to as they work on assignments.

\begin{figure}[ht]
    \centering
    \includegraphics[width=\textwidth]{photos/qf-repository-structure.png}
    \caption{GitHub repository structure for a QuickFeed course}
    \label{fig:qf-repository-structure}
\end{figure}

\subsection{Assignment Submission Process}

To fully understand how a standard assignment submission process usually goes, we will detail it in this section.

As teachers push their assignments to the \textit{assignments} repository, students will pull it to their own repositories.
When students feel compelled, they push their source code to their respective GitHub repository.
This triggers QuickFeed to run the tests that are defined for that assignment in the \textit{tests} repository.
Students can then view the results of these tests through QuickFeed's web based user interface.  
This allows students to see which parts of their code failed, if any.

If an assignment is automatically graded, they will receive a passing grade once enough tests pass.
If it is instead set to be manually graded, a teacher must at some point explicitly approve the assignment.

\subsection{Webhooks}

QuickFeed communicates with GitHub in two ways.
In this section we will detail one of them, namely webhooks.

A webhook is a "user-defined callback over HTTP". % (ref: (09.04) https://developer.atlassian.com/server/jira/platform/webhooks/)
In general, webhooks allow developers to listen to events from any supporting sites.
When any such event occurs, an HTTP request is sent to the address configured for the webhook.
The request contains data about the event, usually in a JSON format.

QuickFeed uses webhooks to retrieve data from push events on a course.
When a new course is created, QuickFeed creates a webhook on the GitHub organization of the given course.
This webhook is only triggered by push events, which are then handled by QuickFeed accordingly:

\begin{itemize}
    \item If the push event is to the tests repository, QuickFeed will update the course assignments.
    \item If the push event is to a student repository, QuickFeed will determine the assignments that have been changed/worked on, 
    and run the assignment tests on them.
\end{itemize}

\subsection{SCM API}

The second way QuickFeed communicates with GitHub is through a custom API.

To facilitate communication with potentially any source control management system, a custom SCM API has been developed for QuickFeed.
The current iteration of QuickFeed supports interacting with GitHub through this API, via the go-github library. % (ref: go-github)
The library itself, communicates with GitHub's own REST API using HTTP requests.

The SCM API allows QuickFeed to perform various tasks, such as creating/updating repositories, creating webhooks, managing teams and more.
It, together with webhooks, therefore form a 2-way communication stream between QuickFeed and GitHub.