%!TEX root = ../thesis.tex

\chapter{Approach}
\label{ch:approach}
% TODO: Revise
In this chapter we discuss how this project was approached.

Though the project itself is fairly straight forward, how to approach it became quite challenging.
Planning out how to implement the desired features in a good and approachable manner, would prove highly time consuming and hard.
Even the desired features themselves were hard enough to conceptualise.
A problem that was not helped by the fact that there was no clear single source of information to describe them.
As a consequence, the approach to this project was always changing, as new desires and ideas were discussed.

In this chapter we discuss this ever changing approach.
How it started as a continuation of previous approaches and a desired workflow.
And how it eventually was altered to combat numerous challenges and issues that were faced.
 
\section{Existing Approach}
% TODO: Replace this with overview chapter on individual implementation sections
An implementation for this project has already been attempted by Adil Khurshid. % TODO: Ref
This section discusses the approach he used.

In short, he proposed implementing support for the following features.
When a teacher creates an assignment, they have the option of creating any number of task markdown files.
For each of these markdown files, a single issue should be created on every student repository.
The title and body of these issues are defined by the contents of the corresponding task.
When a student wants to complete a task, they create a pull request for the associated issue.
After the deadline of the assignment has passed, QuickFeed randomly assigns co-student and teacher reviewers to the pull request. 
Only when the pull request has been approved by a teacher, should it be merged.

Adil discusses several challenges to this approach, and gives his proposed solutions.
Given that he did not finish his work however, many of these solutions have later proved inadequate or faulty, and in some cases potentially too complex.
In general, this thesis is independent of the aforementioned approach.

\section{Desired workflow}

This section describes in detail the desired student and teacher workflow, and is in essence what we want to achieve with this project.

Tasks will be handled as described in the previous section.
They are created by the teacher as markdown files within an assignment, and QuickFeed creates issues based on them.
When a student wants to complete any task, they create a new branch on their local repository.
Using this branch, they create a pull request on their student repository which is linked to the associated issue.

At this stage, a student has an open pull request on their repository.
Tests are run on the student code by QuickFeed, and students receive feedback on these tests in the pull request itself.

When students are finished with a task, i.e., they receive a passing score from the test results, their code is ready to be reviewed.
At this point, QuickFeed should automatically assign one student and teacher reviewer.
They will comment on the students code, request changes and so on.
When all changes are implemented, the teacher approves the pull request, allowing the student to merge it.
Finally, when all tasks have gone through this process, the assignment itself can be set as approved.

\section{Access to Student Repositories}

In the previous section we described what we want to achieve.
Beginning with this section, we start discussing some initial challenges.

\subsection{The Problem}

In order for one student to review the pull request of another, they will have to be given access to it.

A solution could be to use GitHub teams to grant students review access to other repositories.
Access can then be removed once the review process is finished.
This solution does however lead to possible complications.
First of all, students having access to another random student's repository can be somewhat invasive.
Secondly it can lead to the following possible scenario:

Imagine we have two students: Student-1 and Student-2.
Student-2 is finished with a task from assignment1, and therefore gets a reviewer assigned.
Student-1 is assigned to review it, and is granted access by QuickFeed to Student-2's repository.
Student-2 is also diligent, and has already started working on assignment2.
In fact, Student-2 has nearly completed assignment2.
Student-1 however, has not even started on assignment2, but because they can now access Student-2's assignments, they receive valuable knowledge on how to solve it.
Of course, this would be highly problematic to any teacher wanting their students to solve assignments independently.

To avoid this problem, a possible solution is to create a clone repository, containing only the assignment that is to be reviewed. 
This way, any reviewing student will only have access to the assignment in question, and not the rest of that students repository.

% TODO: Revise this example.
Implementing this however seems highly complex, as there is a myriad of complications and problems that will have to be accounted for.
For one, if an assignment contains for example 6 tasks, it would mean that 6 copy repositories will have to be created and managed by QuickFeed for each student.
If a course has 50 enrolled students, 300 new repositories will then have to be created, for just that one assignment.
Obviously this is not a feasible way to tackle this issue.

Even if we manage to reduce the number of such repositories that need to be created, there is still the problem of how to manage them.
An implementation needs to handle, amongst other things, the following:

\begin{itemize}
    \item How to create new review repositories containing only the code that is relevant.
    \item How to delete them when they are no longer needed.
    \item How to associate actions between a review repository and the original student repository.
          E.g. if code is updated in the original, should it then also be updated in the other?
          And if so, how do we accomplish this?
    \item How to manage pull requests in two different repositories.
\end{itemize}

Let us say that we, despite these challenges, managed to create a fully functional implementation of the above solution.
We now have to cope with the prospect that we have further complexified the user experience.
Students having to learn GitHub pull requests may be problematic enough on its own, but also introducing them to review repositories could prove counter-productive.

\subsection{Limiting Scope to Group Repositories}

The two solutions proposed in the previous section both seem suboptimal.
Though they are possible to implement, doing so in a functional manner seems difficult and time consuming.
Instead of going through with one of them, a compromise was agreed on to limit the project scope to only group assignments.

Instead of students reviewing each other's code on regular assignments, they will only do so on group assignments.
Furthermore, students will only review other group members' code.
As an example, we can imagine a group assignment with three different tasks.
A group of three students will create a pull request each to solve these tasks.
They are then each assigned, when appropriate, to review one of the other group members' pull request

The benefit of this approach is that it avoids students needing access to other repositories all together.
We only need to manage student review within a local group scope, and not for an entire course.

\section{Automated Student Feedback}

The initial idea was to use workflows.
Does not seem possible using GitHub workflows. 
Seems like we must find an alternative solution.
Also what should this feedback consist of?

\subsection{GitHub Checks API}

\subsection{Pull request comments}

\subsection{Commit Status API}